\section{Formulation}
\label{formulation}

A detailed formulation will be described here. \cite{allen:1985} is a
classical reference describing the mathematical model.

\subsection{Steady-state Darcy flow}
\label{darcy-flow}

This program solves pressure-driven, steady-state Darcy's flow on a
square plate with spatially varying permeability.

\begin{equation}
\begin{split}
\bK^{-1}\bu + \grad(p) & = 0\\
\div(\bu) & = 0
\end{split}
\end{equation}

Which, in weak form reads:

\begin{equation}
\begin{split}
(\bv, \bK^{-1}\bu) - (\div(\bv), p) = & - (\bv, p\bn)_N \quad\forall v \\
(q, \div(\bu)) = & 0 \quad\forall q
\end{split}
\end{equation}

It then proceeds to construct and solve the adjoint problem, with a
suitable goal functional of interest as its right hand-side.

\begin{equation}
(\bw, \bK^{-1}\bv) - (\div(\bw), q) + (r, \div(\bv)) = M(v) \quad\forall
v, q
\end{equation}

We solve this adjoint problem for the variable \bz{} = (\bw, r). After
this, the error is estimated (rather crudely for now: ignoring jump
terms and constant multipliers) using the dual weighted residual
method:

\begin{equation}
M(\bu) - M(\bu_h) \approx  \sum_T |<\bR_T, \bz - \bz_h>_T + <\bR_{\partial T}, \bz - \bz_h>_{\partial T}|
\end{equation}

The error estimate is used to suitably refine the mesh, and the above
process is repeated until a certain tolerance is reached.

\subsection{Two-phase flow}
\label{two-phase-flow}

We are interested in the flow of two immiscible, incompressible fluids
flowing through a heterogeneous porous medium, $\Omega \subset
\mathbb{R}^{d}\; (d = 2, 3)$ over the time interval $\mathcal{I} = [0,
T]$. In strong form $(S)$, our model problem for this case reads: Find
the total velocity $\bu: \Omega \times \mathcal{I} \rightarrow
\mathbb{R}^{d}$, pressure $p: \Omega \times \mathcal{I} \rightarrow
\mathbb{R}$ and saturation $s: \Omega \times \mathcal{I} \rightarrow
\mathbb{R}$ such that

\begin{equation}
  \begin{split}
    (\lambda(s)\bK)^{-1}\bu + \grad(p) & = 0\\
    \div(\bu) & = 0\\
    \frac{\partial s}{\partial t} + \bu\cdot\grad(F(s)) & = 0,
  \end{split}
\label{two-phase-flow-strong}
\end{equation}

\noindent where $\bK: \Omega \rightarrow \mathbb{R}^{d \times d}$ is
a given spatially-varying permeability tensor, $\lambda(s)$ is the
total mobility and $F(s)$ is the fractional flow function. The
treatment presented in this paper is general and independent of the
functional forms of these quantities, but in order to fix concepts, we
choose:
\begin{equation}
  \begin{split}
    \lambda(s) & = \frac{1}{\mu_{\mathrm{rel}}} s^{2} + (1 - s)^{2}
    \quad \mathrm{and}\\
    F(s) & = \frac{s^{2}}{s^{2} + \mu_{\mathrm{rel}} (1 - s)^{2}},
  \end{split}
\end{equation}

\noindent for some constant relative viscosity
$\mu_{\mathrm{rel}}$.\footnote{Refer \cite{allen:1985} for details
  relating to the continuum physics underlying this formulation.}

Along with problem~(\ref{two-phase-flow-strong}), we specify the
following compatible initial conditions:
\begin{equation}
  \begin{split}
    \bu(\bx, 0) & = \bu_{0}(\bx)\; \mathrm{in}\; \Omega\\
    p(\bx, 0) & = p_{0}(\bx)\; \mathrm{in}\; \Omega\\
    s(\bx, 0) & = s_{0}(\bx)\; \mathrm{in}\; \Omega,
  \end{split}
\end{equation}
and boundary conditions:
\begin{equation}
  \begin{split}
    \bu & = \bg_{u}\; \mathrm{on}\; \partial \Omega_{D_{u}}\\
    p & = g_{p}\; \mathrm{on}\; \partial \Omega_{D_{p}}\\
    s & = g_{s}\; \mathrm{on}\; \partial \Omega_{D_{s}}
    \quad \mathrm{and}
  \end{split}
\end{equation}
\begin{equation}
  \begin{split}
    p \bn & = \overline{p \bn}\; \mathrm{on}\; \partial \Omega_{N_{u}}\\
    \bu \cdot \bn & = \overline{\bu \cdot \bn}\;
    \mathrm{on}\; \partial \Omega_{N_{p}}\\
    F(s) \bu \cdot \bn & = \overline{F(s) \bu \cdot \bn}\;
    \mathrm{on}\; \partial \Omega_{N_{s}}.
  \end{split}
\end{equation}

When reposed in weak form $(W)$ pertinent to numerical implementation
by the finite element method, problem~(\ref{two-phase-flow-strong})
reads: Find $(\bu, p, s) \in W = V \times Q \times R$, such that
\begin{equation}
a((\bv, q, r); (\bu, p, s)) = L((\bv, q, r))
\end{equation}
for all $(\bv, q, r) \in \hat{W} =  \hat{V} \times \hat{Q} \times
\hat{R}$,
where
\begin{equation}
\begin{split}
a((\bv, q, r); (\bu, p, s)) = &\phantom{-}
    \int_{0}^{T} \langle \bv,\, (\lambda(s) K)^{-1} \bu \rangle \, \mathrm{d}t\;
  - \int_{0}^{T} \langle \div(\bv),\, p \rangle \, \mathrm{d}t\;
- \int_{0}^{T} \langle \grad(q),\, \bu \rangle \, \mathrm{d}t\\
& + \int_{0}^{T} \langle r,\, \frac{\partial s}{\partial t} \rangle \, \mathrm{d}t\;
  - \int_{0}^{T} \langle \grad(r),\, F(s) \bu \rangle \, \mathrm{d}t
  \quad \mathrm{and}
\end{split}
\end{equation}
\begin{equation}
\begin{split}
L((\bv, q, r)) =
& - \int_{0}^{T} \langle \bv,\, \overline{p \bn} \rangle_{\partial
  \Omega_{N_{u}}} \, \mathrm{d}t
 - \int_{0}^{T} \langle q,\, \overline{\bu \cdot \bn}
\rangle_{\partial \Omega_{N_{p}}} \, \mathrm{d}t
 - \int_{0}^{T} \langle r,\, \overline{F(s) \bu \cdot \bn}
\rangle_{\partial \Omega_{N_{s}}} \, \mathrm{d}t.
\end{split}
\end{equation}

%\noindent where $\bx$ denotes a point in $\Omega$, and $t$ denotes a
%time instant in the interval $\mathcal{I}$.

% \noindent And upon computing the total velocity, $\bu$, one can then
% can post-calculate the velocity of each phase using the relation:
% $\bu_{j} = - (k_{rj}(s)/\mu_{j})\, \bK\, \grad(p)$.

%& = k_{rw}(s)/\mu_w/(k_{rw}(s)/\mu_w + k_{ro}(s)/\mu_o) \\

{\bf Weak form:}

Find $\bu, p, s \in V$ such that,

\begin{equation}
\begin{split}
(\bv, (\lambda\bK)^{-1}\bu) - (\div(v), p) = & - (\bv, \bar{p}\bn)_N \\
                           (q, \div(\bu)) = & 0\\
(r, \frac{\partial s}{\partial t}) - (\grad(r), F\bu) = & - (r,
F\bu\cdot\bn)_N
\end{split}
\end{equation}

$\forall \bv, q, r \in \hat{V}.$

{\bf Adjoint problem:}

Find $\bz_u, z_p, z_s in \hat{V}$ such that,

\begin{equation}
(\bz_u, (\lambda\bK)^{-1}\bv) - (\div(\bz_u), q) + (z_p, \div(\bv))
+ (z_s, r) - (z_s, s0) - dt (\grad(z_s), F\bv) + dt (z_s,
F\bv\cdot\bn)_N = M(\bv)
\end{equation}

$\forall \bv, q, r \in V - V_h.$

{\bf Error estimate (per time step):}

Using the dual weighted residual method (ignoring jump terms and and
constant multipliers),

\begin{equation}
M(\bu) - M(\bu_h) \approx  \sum_T |<\bR_T, \bz - \bz_h>_T +
<\bR_{\partial T}, \bz - \bz_h>_{\partial T}| .
\end{equation}

% Model problem:

%  -----4-----
%  |         |
%  1         2
%  |         |
%  -----3-----

% Initial conditions:
% u(x, 0) = 0
% p(x, 0) = 0
% s(x, 0) = 0 in \Omega

% Boundary conditions:
% p(x, t) = 1 - x on \Gamma_{1, 2, 3, 4}
% s(x, t) = 1 on \Gamma_1 if u.n < 0
% s(x, t) = 0 on \Gamma_{2, 3, 4} if u.n > 0

% Goal functionals:
% M(v) = inner(grad(u_h), grad(v))*dx
% M(v) = inner(u_h, v)*dx
% M(v) = inner(v, n)*ds(2)

% Parameters:
% mu_rel, Kinv, lmbdainv, F, dt, T

% This implementation includes functional forms from the deal.II demo
% available at: http://www.dealii.org/6.2.1/doxygen/deal.II/step_21.html

% Local Variables:
% TeX-master: "adaptive-porous-flow"
% mode: latex
% mode: flyspell
% End:
